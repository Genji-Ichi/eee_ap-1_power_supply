% !TEX root = 1_power_supply.tex
\documentclass[1_power_supply.tex]{subfiles}
\graphicspath{{../figures/}}
\begin{document}

\section{DC-DCコンバータ}

\subsection{目的}

太陽電池から得られる直流の電圧,電流を増幅するためにDC-DCコンバータ農地,チュックコンバータを用いる.本実験では,チュックコンバータのスイッチ制御を変化させたときの電力利得を調べる.

\subsection{原理}

DC-DCコンバータは,スイッチングを行うことで電圧を増幅できる装置である.1周期のスイッチングの中でのオンの時間の割合(デューティ比 $D$)によって,入力電圧$V_\mathrm{i}$と出力電圧$V_\mathrm{o}$との間に
\begin{align}
	V_\mathrm{o} = -\frac{D}{1-D}V_\mathrm{i}
\end{align}

という関係が成り立つ.

\subsection{方法}

\begin{figure}[htbp]
	\begin{center}
		\scalebox{0.1}{\includegraphics{1_4.png}}
		\caption{測定対象の回路}\label{fig:1_4}
	\end{center}
\end{figure}

測定対象の回路を図\ref{fig:1_4}に示す.この回路に対して,方形波にデューティ比がそれぞれ \\$\SIs{10}{\percent},\SIs{25}{\percent},\SIs{40}{\percent},\SIs{50}{\percent},\SIs{60}{\percent},\SIs{75}{\percent},\SIs{80}{\percent}$を印加したときの入力電圧$V_\mathrm{i}$,入力電流$I_\mathrm{i}$,出力電圧$V_\mathrm{o}$,出力電流$I_\mathrm{o}$を測定し,入力電力$P_\mathrm{i}$と出力電力$P_\mathrm{o}$の関係をプロットする.
	各素子や信号の詳細は,
	\begin{enumerate}
		\item $L_1=\SIs{470}{\text{\textmu}\henry}$ % \micro が使えなかったのでやむを得ず
		\item $L_2=\SIs{470}{\text{\textmu}\henry}$
		\item $C_1=\SIs{100}{\text{\textmu}\farad}$
		\item $C_2=\SIs{100}{\text{\textmu}\farad}$
		\item 方形波 : (周波数 : $\SIs{100}{kHz}$, 電圧 : $\SIs{5}{\vpp}$, オフセット : $\SIs{2.5}{\volt}$)
	\end{enumerate}
	である.

	\subsection{使用器具}

	(型番については確認し次第追記します.)
	\begin{enumerate}
		\item 電圧計
		\item 電流計
		\item 直流電源
		\item ファンクションジェネレータ
	\end{enumerate}

	\subsection{結果}

	各デューティ比$D$での電力利得を図\ref{fig:2_10p},\ref{fig:2_25p},\ref{fig:2_40p},\ref{fig:2_50p},\ref{fig:2_60p},\ref{fig:2_75p},\ref{fig:2_80p},に示す.

	\begin{figure}[htbp]
		\begin{minipage}{0.45\columnwidth}
			\centering
			\includegraphics[width=0.8\columnwidth]{2_10p.pdf}
			\caption{$D=0.1$}\label{fig:2_10p}
		\end{minipage}
		\begin{minipage}{0.45\columnwidth}
			\centering
			\includegraphics[width=0.8\columnwidth]{2_25p.pdf}
			\caption{$D=0.25$}\label{fig:2_25p}
		\end{minipage}

		\vspace{1.5mm}
		\begin{minipage}{0.45\columnwidth}
			\centering
			\includegraphics[width=0.8\columnwidth]{2_40p.pdf}
			\caption{$D=0.4$}\label{fig:2_40p}
		\end{minipage}
		\begin{minipage}{0.45\columnwidth}
			\centering
			\includegraphics[width=0.8\columnwidth]{2_50p.pdf}
			\caption{$D=0.5$}\label{fig:2_50p}
		\end{minipage}

		\vspace{1.5mm}
		\begin{minipage}{0.45\columnwidth}
			\centering
			\includegraphics[width=0.8\columnwidth]{2_60p.pdf}
			\caption{$D=0.6$}\label{fig:2_60p}
		\end{minipage}
		\begin{minipage}{0.45\columnwidth}
			\centering
			\includegraphics[width=0.8\columnwidth]{2_75p.pdf}
			\caption{$D=0.75$}\label{fig:2_75p}
		\end{minipage}

		\vspace{1.5mm}
		\begin{minipage}{0.45\columnwidth}
			\centering
			\includegraphics[width=0.8\columnwidth]{2_80p.pdf}
			\caption{$D=0.8$}\label{fig:2_80p}
		\end{minipage}
		\begin{minipage}{0.45\columnwidth}
			\centering
			\includegraphics[width=0.8\columnwidth]{2_d_eta.pdf}
			\caption{デューティ比と電力利得の関係}\label{fig:2_d_eta}
		\end{minipage}
	\end{figure}

	\subsection{考察}

	デューティ比と電力利得の関係を表すプロットを図\ref{fig:2_d_eta}に示す.この関係から,電力利得を良くするためにはデューティ比が$0.4\sim 0.7$の範囲で動作させればいいことがわかる.
