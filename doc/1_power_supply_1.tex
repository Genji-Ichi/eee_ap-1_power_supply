% !TEX root = 1_power_supply.tex
\documentclass[1_power_supply.tex]{subfiles}
\graphicspath{{../figures/}}
\begin{document}

\section{太陽電池の特性}

  \subsection{目的}

    \begin{enumerate}
      \item 照度と,光源からの距離との関係を調べる.
      \item 太陽電池の特性を測定し,等価回路のパラメータや出力できる電力について評価する.
    \end{enumerate}

  \subsection{原理}

    今回作成するシステムの入口となる光エネルギーから電気エネルギーへの変換部分を担うのが太陽電池であり,これは,ダイオードの光特性を利用した素子である.太陽電池の等価回路は図\ref{fig:1_2}の青点線部で表される.このとき図\ref{fig:1_2}中の$V,I$に関して,
    \begin{align}
      I = I_\mathrm{ph}-I_0\left\{\exp[\frac{e(V+R_\mathrm{s}I)}{nkT}-1]\right\}-\frac{V+R_\mathrm{s}I}{R_\mathrm{sh}}
    \end{align}

    という関係が成り立つ.ここで,$I_0$ :飽和電流,$e$ :電気素量,$k$ :ボルツマン定数,$n$ :接合定数,$T$ :絶対温度,$I_\mathrm{ph}$ :光電流である.


    \begin{figure}[htbp]
      \begin{center}
        \scalebox{0.1}{\includegraphics{1_2.png}}
        \caption{太陽電池の特性を測定するための回路}\label{fig:1_2}
      \end{center}
    \end{figure}

    等価回路の抵抗を求める際は,ダイオードが動作している(電圧源とみなせる)ときと,動作していない(開放とみなせる)ときに分けて考える.

    \subsubsection{ダイオードが動作しているとき}

      ダイオードの動作電圧を$V_\mathrm{on}$とすると,キルヒホッフの電圧則から,

      \begin{align}
        V_\mathrm{on} = V+R_\mathrm{s}I
      \end{align}

      両辺を$V$で微分して,

      \begin{align}
        0            &= 1+R_\mathrm{s}\dv{I}{V}  \\
        R_\mathrm{s} &= \left.- \dv{V}{I}\right|_{V\sim V_\mathrm{on}} \label{eq:R_s}
      \end{align}

      となる.

    \subsubsection{ダイオードが動作していないとき}

      キルヒホッフの電圧則から,

      \begin{align}
        V+R_\mathrm{s}I = R_\mathrm{sh}(I_\mathrm{ph}-I)
      \end{align}

      両辺を$I$で微分して,

      \begin{align}
        \dv{V}{I}+R_\mathrm{s} &= -R_\mathrm{sh}  \\
        R_\mathrm{sh}          &= \left.-\dv{V}{I}\right|_{V\sim 0}-R_\mathrm{s} \label{eq:R_sh}
      \end{align}

  \subsection{方法}

    \subsubsection{照度と光源からの距離に関する実験(実験1.1)}


      \begin{figure}[htbp]
        \begin{center}
          \scalebox{0.2}{\includegraphics{1_1.png}}
          \caption{照度測定の様子を横から見た図}\label{fig:1_1}
        \end{center}
      \end{figure}

      図\ref{fig:1_1}のように擬似太陽と太陽光電池,照度計を配置し,疑似太陽と太陽光電池との間隔を変えて,太陽光電池の中央での照度を計測した.測定点は$\mathrm{OB}=\SIs{60}{\centi\meter},\SIs{75}{\centi\meter},\SIs{90}{\centi\meter},\SIs{105}{\centi\meter},\SIs{120}{\centi\meter}$の5点とした.
      % なお,距離の測定は図\ref{fig:1_1}中OB間で行い,適切な変換でもってAC間の距離を導出した.測定した間隔は,OB間が$\SIs{60}{\centi\meter},\SIs{75}{\centi\meter},\SIs{90}{\centi\meter},\SIs{105}{\centi\meter},\SIs{120}{\centi\meter}$の5つの場合である.

    \subsubsection{太陽電池の特性を調べる実験(実験1.2)}


      \begin{figure}[htbp]
        \begin{center}
          \scalebox{0.11}{\includegraphics{1_3.png}}
          \caption{太陽電池の特性を測定するための回路($I$が$\SIs{0}{\ampere}$近傍のとき)}\label{fig:1_3}
        \end{center}
      \end{figure}

      図\ref{fig:1_1}中OB間の距離が$\SIs{60}{\centi\meter},\SIs{105}{\centi\meter},\SIs{120}{\centi\meter}$の3つの場合について,図\ref{fig:1_2}の回路を用いて$V,I$を測定した.また,$I$が$\SIs{0}{\ampere}$近傍の測定については,抵抗を印加しない開放での測定を再現するため,図\ref{fig:1_3}の回路を用いた.

  \subsection{使用器具}
    (型番を記していない器具については確認し次第追記します.)
    \begin{enumerate}
      \item 太陽電池モジュール : 昭和ソーラーエネルギー(株) GT234
      \item 電圧計
      \item 電流計
    \end{enumerate}

  \subsection{結果}

    \subsubsection{実験1.1}

      $\mathrm{AC}=\mathrm{OB}-\SIs{4.5}{\centi\meter}$と照度の測定結果を図\ref{fig:1_dE}に示す.


      \begin{figure}[htbp]
        \begin{center}
          \scalebox{0.5}{\includegraphics{1_dE.pdf}}
          \caption{光源から照度計までの距離と照度との関係}\label{fig:1_dE}
        \end{center}
      \end{figure}

    \subsubsection{実験1.2}

      各間隔での電流電圧特性の測定結果を図\ref{fig:1_VI}に示す.


      \begin{figure}[htbp]
        \begin{center}
          \scalebox{0.75}{\includegraphics{1_VI.pdf}}
          \caption{光を照射したときの太陽電池の電流電圧特性の測定結果}\label{fig:1_VI}
        \end{center}
      \end{figure}

  \subsection{考察}

    \subsubsection{実験1.1}

      光源が点光源の場合は,照度は距離の逆二乗に比例することが知られている.今回の実験では点光源ではないことを考慮し,照度を$E$,距離を$d$とおいて,

      \begin{align}
        E = \frac{A}{ad^2+bd+c}
      \end{align}

      に従うことを仮定する.$A,a,b,c$は定数である.図\ref{fig:1_dE}に対しフィッティングを行ったものを図\ref{fig:1_dE_fit}に示す.


      \begin{figure}[htbp]
        \begin{center}
          \scalebox{0.5}{\includegraphics{1_dE_fit.pdf}}
          \caption{照度測定結果とそのフィッティング}\label{fig:1_dE_fit}
        \end{center}
      \end{figure}

      今回の測定から,

      \begin{align}
        E = \frac{\SIs{1.47e4}{\lumen}}{(\num{5.4e-4})\times d^2+(\SIs{9.2e-3}{\meter})\times d+(\SIs{0.84}{\meter^2})}
      \end{align}

      と係数を決定できた.

    \subsubsection{実験1.2}

      代表として$\mathrm{OB}=\SIs{105}{\centi\meter}$の場合について解析する.


      \begin{figure}[htbp]
        \begin{center}
          \scalebox{0.5}{\includegraphics{1_VI_fit.pdf}}
          \caption{電流電圧特性の測定結果とフィッティング}\label{fig:1_VI_fit}
        \end{center}
      \end{figure}

      ダイオードが動作している領域・動作していない領域でフィッティングした図を図\ref{fig:1_VI_fit}に示す.ここで得られた傾きと式(\ref{eq:R_s})(\ref{eq:R_sh})から,

      \begin{align}
        \SIeq{R_\mathrm{s}}{\per\ohm}  &=      -\frac{1}{\num{-4.48e-2}}  \\
        R_\mathrm{s}                   &\simeq \SIs{22.3}{\ohm}  \\
        \SIeq{R_\mathrm{sh}}{\per\ohm} &=      -\frac{1}{\num{-5.20e-4}}-22.3  \\
        R_\mathrm{sh}                  &\simeq \SIs{1.90e3}{\ohm}
      \end{align}

      と,等価回路の抵抗値が得られる.
